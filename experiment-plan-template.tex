\documentclass[a4paper]{ltjsreport}
\usepackage[top=16truemm,bottom=16truemm,left=14truemm,right=14truemm]{geometry}
\usepackage{enumitem}
\usepackage{fancyhdr}
\usepackage{color}
\usepackage{xcolor}
\usepackage{listings}
\usepackage{lastpage}
\usepackage{hyperref}
\usepackage{enumerate}
\usepackage{tcolorbox}
\tcbuselibrary{breakable, skins}
\usepackage{luacode}
\usepackage{etoolbox}
\usepackage{bxbase}
\usepackage{luatexja-otf}
\usepackage{amsmath}
\usepackage{amssymb}
\usepackage{multicol}
\usepackage{url}

\hypersetup{
  luatex,
 bookmarks=false,
 colorlinks=true,
 linkcolor=black,
 citecolor=[rgb]{0,0.4,0.8},
 filecolor=black,
 urlcolor=[rgb]{0,0.4,0.8}
}


\pagestyle{fancy}
\lhead{\fontsize{8pt}{0pt}\selectfont \hyperlink{top}{実験計画書}}
\rfoot{\href{https://github.com/miko-misa/scitokyo24-experiment-template/}{実験計画書テンプレート}}
\cfoot{\thepage}


\begin{document}
\begin{multicols}{2}
\hspace{-1em}{\fontsize{38pt}{80pt}\selectfont 実験計画書}
\vspace*{5px}
\\
{\large \textbf{【実験テーマ】}No.9 電磁気}\\
\\
\\
\begin{description}[labelwidth=6.0em]
  \setlength{\leftskip}{3.8cm}
  \item[実験日] 2024年4月15日
  \item[曜日-組-班] 月-白-ロ
  \item[学籍番号] 24B00000
  \item[名前] 佐藤 太郎
\end{description}
\end{multicols}

\section*{目的}
ここには目的を書くのです。
\section*{手順}
ここには手順を書くのです。





\end{document}